\documentclass{report}
\usepackage[utf8]{inputenc}
\usepackage{amsmath}
\usepackage{graphicx}

\title{Øvelser - Divide-and-Conquer}
\author{André Oskar Andersen (wpr684)}
\date{\today}

\begin{document}
\maketitle

\section*{12.1 - 5}
\textbf{Argue that since sorting $n$ elements takes $\Omega(n \lg n)$ time in worst case in the comparison model, any comparison-based algorithm for constructing a binary search tree from an arbitrary list of $n$ elements takes $\Omega(n \lg n)$ time in the worst case} \\
Assume, for the sake of contradiction, that we can construct the binary search tree by comparison-based algorithm using less than $\Omega(n \lg n)$ time, since the inorder tree walk is $\Theta(n)$, then we can get te sorted elements in less than $\Omega(n \lg n)$ time, which contradicts the fact that sorting $n$ elements takes $\Omega(n \lg n)$ time in the worst case

\section*{12.2 - 3}
\textbf{Write the \texttt{TREE-PREDECESSOR} procedure}
\begin{verbatim}
TREE-PREDECESSOR(x):
    if x.left != NIL:
        return TREE-MAXIMUM(x.left)
    y = x.p
    while (y != NIL and x == y.left):
        x = y
        y = y.p
    return y
\end{verbatim}

\section*{12.3 - 4}
\textbf{Is the operation of deletion "commutative" in the sense that deleting $x$ and then $y$ fro ma binary search tree leaves the same tree as deleting $y$ and then $x$? Argue why it is or give a counterexample} \\
No, it is not. deleting $A$ then $B$ does not yield the same result, as deleting first $B$ then $A$:
\begin{verbatim}
    A        C        C
    / \      / \        \
   B   D    B   D        D
      /
     C
\end{verbatim}

\begin{verbatim}
    A        A        D
    / \        \      /
   B   D        D    C
      /        /
     C        C   
\end{verbatim}

\section*{13.1 - 2}
\textbf{Draw the red-black tree that results after \texttt{TREE-INSERT} is called on the tree in figure 13.1 with key 36. If the inserted node is colered red, is the resulting tree a red-black tree? What if it is colored black?}
Indsættes knuden som rød, bliver egenskab 4 ikke. Indsættes knuden som sort, overholdes egenskab 5 ikke. Træet tegnes ved, at der sættes en knude som højrebarn til knuden med $key = 35$

\section*{13.2 - 3}
\textbf{Let $a$, $b$, and $c$ be arbitrary nodes in subtrees $\alpha$, $\beta$, and $\gamma$, respectively, in the left tree of Figure 13.2. How do the depths of $a$, $b$, and $c$ change when a left rotation is performed on node $x$ in the figure?} \\
$\alpha$'s dybde bliver én større, $\beta$ bliver som den er og $\gamma$ bliver én mindre 

\section*{13.3 - 1}
\textbf{In line 16 of \texttt{RB-INSERT}, we set the color of the newly inserted node $z$ to red. Observe that if we had chosen to set $z$'s color to black, then property 4 of a red-black tree would not be violated. Why didn't we choose to set $z$'s color to black?} \\
Hvis $z$'s farve var sort, ville egenskab 5 ikke bliver overholdt.



\end{document}