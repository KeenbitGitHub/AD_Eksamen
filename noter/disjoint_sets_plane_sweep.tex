\documentclass{report}
\usepackage[utf8]{inputenc}
\usepackage{amsmath}
\usepackage{amssymb}
\usepackage{graphicx}

\title{Eksamensnoter - Disjoint Sets and Plane Sweep}
\author{André Oskar Andersen (wpr684)}
\date{\today}

\begin{document}
\maketitle

\section*{21 Data Structures for Disjoint Sets}
\begin{itemize}
    \item Some applications involve grouping $n$ distinct elements into acollection of disjoint sets. These applications often need to perform two operations in particular: finding the unique sets that contains a given element and uniting two sets.
\end{itemize}

\subsection*{21.1 Disjoint-set operations}
\begin{itemize}
    \item A \textit{disjoint-set data structure} maintains a collection $\mathcal{S} = \{S_1, S_2, ..., S_k\}$ of disjoint dynamic sets. We identify each set by a \textit{representative}, which is some member of thes et. If we ask for the representative of a dynamic set twice without modifying the set between the requests, we get the same anaswer both times.
    \item We represent each element of a set by an object. Letting $x$ denote an object, we wish to support the following operations:
    \begin{itemize}
        \item \texttt{MAKE-SET($x$)}: creates a new set whose only member (and thus representative) is $x$. Since the sets are disjoint, we require that $x$ not already be in some other set
        \item \texttt{UNION($x$, $y$)}: unites the dynamic sets that contain $x$ and $y$, into a new set that is the union of these two sets.
        \item \texttt{FIND-SET($x$)}: returns a pointer to the representative of the (unique) set containg $x$
    \end{itemize}
    \item Since the sets are disjoint, each \texttt{UNION} operation reduces the number of sets by one. After $n - 1$ \texttt{UNION} operations, therefore, only one set remains. The number of \texttt{UNION} operations is thus at most $n - 1$.
\end{itemize}
\textbf{An application of disjoint-set data structures}
\begin{itemize}
    \item One of the many applications of disjoint-set data structures arises in determinening the conencted components of an undirected graph.
    \item The procedure \texttt{CONNECTED-COMPONENTS} that follows uses the disjoint-set operations to compute the connected components of a graph. Once \texttt{CONNECTED-COMPONENTS} has preprocessed the graph, the procedure \texttt{SAME-COMPONENTS} answers queries about whether two vertices are i nthe same connected component.
\end{itemize}

\subsection*{21.2 Linked-list representation of disjoint sets}
\begin{itemize}
    \item A simple way to implement a disjoint-set data structure: each set is represented by its own linked list. The object for each set has attributes \textit{head}, pointing to the first object i nthe list, and \textit{tail}, pointing to the last object. Each object in the list contains a set member, a pointer to the next object in the list, and a pointer back to the set object. Within each linked list, the objects may appear in any order. The representative is the set member in the first object in the list.
    \item Both \texttt{MAKE-SET} and \texttt{FIND-SET} requires $O(1)$ time.
    \item To carry out \texttt{MAKE-SET($x$)}, we create a new linked list whose only object is $x$
    \item For \texttt{FIND-SET($x$)}, we just follow the pointer from $x$ back to its set object and the nreturn the member in the object that \textit{head} points to.
\end{itemize}
\textbf{A simple implementation of union}
\begin{itemize}
    \item The simplest implementation of the \texttt{UNION} operation using the linked-list set representation takes significantly more time than \texttt{MAKE-SET} or \texttt{FIND-SET}. We perform \texttt{UNION($x$, $y$)}, by appending $y$'s list onto the end of $x$'s list. The reåresentative of $x$'s list becomes the representative of the resulting set.
\end{itemize}
\subsection*{21.3 Disjoint-set forests}
\begin{itemize}
    \item In a faster implementation of disjoint sets, we represent sets by rooted trees, with each node containing one member and each tree representing one set.
    \item In a \textit{disjoint-set forest} each member points only to its parent. The root of each tree contains the representative and is its own parent.
    \item A \texttt{MAKE-SET} operations simply creates a tree with just one node. We perform a \texttt{FIND-SET} operation by following parent pointers until we find the root of the tree. A \texttt{UNION} operation causes the root of one tree to point to the root of the other
\end{itemize}

\end{document}