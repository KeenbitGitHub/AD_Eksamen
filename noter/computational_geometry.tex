\documentclass{report}
\usepackage[utf8]{inputenc}
\usepackage{amsmath}
\usepackage{amssymb}
\usepackage{graphicx}

\title{Eksamensnoter - Computational Geometry}
\author{André Oskar Andersen (wpr684)}
\date{\today}

\begin{document}
\maketitle

\section*{33 Computational Geometry}
\begin{itemize}
    \item Computational geometry is the branch of computer science that studies algorithms, for solving geometric problems
    \item The input to a computational-geometry problem is typically a description of a set of geometric objects, such as a set of points, a set of line segments, or the vertices of a polygon in counterclockwise order
    \item The output is often a response to a query about the objects, such as whether any of the lines intersect, or perhaps a new geometric object, such as the convex hull of the set of points
    \item In this chapter, we look at a few computational-geometry algorithms in two dimensions. We represent each input object by a set of points $\{p_1, p_2, p_3,...\}$, where $p_i = (x_i, y_i)$ and $x_i, y_i \in \mathbb{R}$
\end{itemize}
\subsection*{33.1 Line-segment properties}
\begin{itemize}
    \item A \textit{convex combination} of two distinct points $p_1 = (x_1, y_1)$ and $p_2 = (x_2, y_2)$ is any point $p_3 = (x_3, y_3)$ such that for some $\alpha$ in the range $0 \leq \alpha \leq 1$, we have $x_3 = \alpha x_1 + (1 - \alpha)x_2$ and $y_3 = \alpha y_1 + (1 - \alpha) y_2$. We also write that $p_3 = \alpha p_1 + (1 - \alpha) p_2$. Intuitively, $p_3$ is any point that is on the line passing through $p_1$ and $p_2$ and is on or between $p_1$ and $p_2$ on the line.
    \item Given two distinct points $p_1$ and $p_2$, the \textit{line segment} $\bar{p_1p_2}$ is the set of convex combinations of $p_1$ and $p_2$
    \item We call $p_1$ and $p_2$ the \textit{endpoints} of segment $\bar{p_1p_2}$.
    \item Sometimes the ordering of $p_1$ and $p_2$ matters, and we speak of the \textit{directed segment} $\overrightarrow{p_1p_2}$
    \item If $p_1$is the origin, then we can treat the directed segment $\overrightarrow{p_1p_2}$ as the vector $p_2$
\end{itemize}
\textbf{Cross products}
\begin{itemize}
    \item We can interpret the \textit{cross product} $p_1 \times p_2$ as the signed area of the parallelogram formed by the points $(0, 0), p_1, p_2$ and $p_1 + p_2 = (x_1 + x_2, y_1 + y_2)$. An equivalent definition gives the cross product as the determinant of a matrix:
    \[
    p_1 \times p_2 =
    \det
    \begin{pmatrix}
       x_1 & x_2 \\
       y_1 & y_2 
    \end{pmatrix}
    =
    x_1 y_2 - x_2 y_1
    = -p_2 \times p_1    
    \]
    \item if $p_1 \times p_2$ is positive, then $p_1$ is clockwise from $p_2$, with respect to the origin $(0, 0)$; if this cross product is negative, then $p_1$ is counterclockwise from $p_2$. A boundary condition arises if the cross product is 0; in this case, the vectors are \textit{colinear}, pointing in either the same or opposite directions
    \item To determine whether a directed segment $\overrightarrow{p_0p_1}$ is closer to a directed segment $\overrightarrow{p_0p_2}$ in a clockwise direction or in a counterclockwise direction with respect to their common endpoint $p_0$, we simply translate to use $p_0$ as the origin. That is, we let $p_1 - p_0$ denote the vector $p'_1 = (x'_1, y'_1)$, where $x'_1 = x_1 - x_0$ and $y'_1 = y_1 - y_0$, and we define $p_2 - p_0$ similarly. We then compute the cross product $(p_1 - p_0) \times (p_2 - p_0) = (x_1 - x_0)(y_2 - y_0) - (x_2 - x_0)(y_1 - y_0)$. If this cross product is positive, then $\overrightarrow{p_0p_1}$ is clockwise from $\overrightarrow{p_0p_2}$; if negative, it is counterclockwise.
\end{itemize}
\textbf{Determining whether consecutive segments turn left or right}
\begin{itemize}
    \item Our next question is whether two line segments $\overline{p_0p_1}$ and $\overline{p_1p_2}$ turn left or right at point $p_1$. Equivalently, we want a method to determine which way a given angle $\angle p_0p_1p_2$ turns.
    \item We simply check whether directed segment $\overrightarrow{p_0p_2}$ is clockwise or counterclockwise relative to directed segment $\overrightarrow{p_0p_1}$. To do so, we compute the cross product $(p_2 - p_0) \times (p_1 - p_0)$. If the sign of this cross product is negative, then $\overrightarrow{p_0p_2}$ is counterclockwise with respect to $\overrightarrow{p_0p_1}$, and thus we make a left turn at $p_1$. A positive cross product indicates a clockwise orientation and a right turn. A cross product of 0 emeans that points $p_0, p_1, \text{ and } p_2$ are colinear. 
\end{itemize}
\textbf{Determining whether two lines segments intersect}
\begin{itemize}
    \item To determine whether two line segments intersect, we check whether each segment straddles the line containing the other. A segment $\overline{p_1p_2}$ \textit{straddles} a line if point $p_1$ lies on one side of the line and point $p_2$ lies on the other side. A boundary case arises if $p_1$ or $p_2$ lies directly on the line. Two line segments intersect if and only if either (or both) of the following conditions holds:
    \begin{enumerate}
        \item Each segment straddles the line containing the other
        \item An endpoint of one segment lies on the other segment (comes from the boundary case)
    \end{enumerate}
\end{itemize}
\subsection*{Determining whether any pair of segments intersects}
\begin{itemize}
    \item This section presents an algorithm for determining whether any two line segments in a set of segments intersect
    \item The algorithm runs in $O(n \lg n)$ time, where $n$ is the number of segments we are given. It determines only whether or not any intersection exists; it does not print all the intersections.
    \item In \textit{sweeping}, an imaginary vertical \textit{sweep line} passes through the given set of geometric objects, usually from left to right. The line-segment-intersection algorithm in thi ssection considers all the line-segment endpoints in the left-to-right order and checks for an intersection each time it encounters an endpoint.
\end{itemize}
\textbf{Ordering segments}
\begin{itemize}
    \item We can order the segments that intersect a verical sweep line according to the $y$-coordinates of the points of intersection
    \item Consider two segments $s_1$ and $s_2$. We say that these segments are \textit{comparable} at $x$ if the vertical sweep line with $x$-coordinate $x$ intersects both of them. We say that $s_1$ is \textit{above} $s_2$ at $x$, written $s_1 \succeq s_2$, if $s_1$ and $s_2$ are comparable at $x$ and the intersection of $s_1$ with the sweep line at $x$ is higher than the intersection of $s_2$ with the same sweep line, or if $s_1$ and $s_2$ intersect at the sweep line.
    \item For a given $x$, the relation $"\succeq_x"$ is a total preorder for all segments that intersect the sweep line at $x$. That is, if segments $s_1$ and $s_2$ each intersect the sweep line at $x$, then either $s_1 \succeq s_2$ or $s_2 \succeq s_1$, or both
\end{itemize}
\textbf{Moving the sweep line}
\begin{itemize}
    \item Sweeping algorithms typically manage two sets of data:
    \begin{enumerate}
        \item The \textit{sweep-line status} gives the relationships among the objects that the sweep line intersects
        \item The \textit{event-point schedule} is a sequence of points, called \textit{event points}, which we order from left to right according to their \textit{x}-coordinates. As the sweep progresses from left to right, wehnever the sweep line reaches the \textit{x-coordinate} of an event point, the sweep halts, processes the event point, and the resumes. Changes to the sweep-line status occur only at event points
    \end{enumerate}
    \item Each segment endpoint is an event point. We sort the segment endpoints by increasing $x$-coordinate and proceed from left to right. When we encounter a segment's left endpoint, we insert the segment into the sweep-line status, and we delete the segment from the sweep-line status upon encountering its right endpoint. Whenever two segments first become consecutive in the total preorder, we check whether they intersect
\end{itemize}
\subsection*{33.4 Finding the closest pair of points}
\begin{itemize}
    \item Consider the problem of finding the closest pair of points in a set \textit{Q} of $n \geq 2$ points.
    \item The divide-and-conquer algorithm for this problem, has a running time described by $T(n) = 2T(n/2) + O(n)$
\end{itemize}
\textbf{The divide-and-conquer algorithm}
\begin{center}
    \includegraphics[width = 10 cm]{../entities/figure_33_11.png}
\end{center}
\begin{itemize}
    \item Each recursive invocation of the algorithm takes as input a subset $P \subseteq Q$ and arrrays $X$ and $Y$, each of which contains all the points of the inpput subset \textit{P}. The points in array \textit{X} are sorted so that their \textit{x}-coordinates are monotonically increasing. Similarly, array \textit{Y} is sorted by monotonically increasing \textit{y}-coordinate.
    \item A given recursive invocation with inputs $P$, $X$, and $Y$ first checks whether $|P| \leq 3$. If so, the invocation simply performs the brute-force method: try all $|P| \choose 2$ pairs of points and return the closest pair. If $|P| > 3$, the recursive invocation carries out the divide-and-conquer paradigm as follows:
    \begin{itemize}
        \item \textbf{Divide}: Find a line \textit{l} that bisects the point set \textit{P} into two equally sized sets $P_L$ and $P_R$. Divide the array \textit{X} into arrays $X_L$ and $X_R$, which contain the points of $P_L$ and $P_R$ respectively, sorted by monotonically increasing \textit{x}-coordinate. Similarly, divide the array \textit{Y} into arrays $Y_L$ and $Y_R$, which contain the points of $P_L$ and $P_R$ respectively, sorted by monotonically increasing \textit{y}-coordinate.
        \item \textbf{Conquer}: Having divided \textit{P} into $P_L$ and $P_R$, make two recursive calls, one to find the closest pair of poitns in $P_L$ and the other to find the closest pair of points in $P_R$. The input to the first call are the subset $P_L$ and arrays $X_L$ and $Y_L$; the second call receives the inputs $P_R$, $X_R$ and $Y_R$. Let the closest-pair distances returned for $P_L$ and $P_R$ be $\delta_L$ and $\delta_R$, respectively, and let $\delta = \min(\delta_L, \delta_R)$
        \item \textbf{Combine}: The closest pair is either the pair with distance $\delta$ found by one of the recursive calls, or it is a pair of points with one point in $P_L$ and the other in $P_R$. If a pair of points has distance less than $\delta$, both points of the pair must be within $\delta$ units of line \textit{l}. Thus, they both must reside in the $2\delta$-wide vertical strip centered at line \textit{l}. To find such a pair, we do the following:
        \begin{enumerate}
            \item Create an array $Y'$, which is the array \textit{Y} with all points not in the $2\delta$-wide vertical strip removed. The array $Y'$ is sorted by \textit{y}-coordinate, just as \textit{Y} is.
            \item For each point \textit{p} in the array \textit{Y'}, try to find points in \textit{Y'} that are within $\delta$ units of \textit{p}. Only the 7 points in \textit{Y'} that follow \textit{p} need to be considered. Compute the distance from \textit{p} to each of these 7 points, and keep track of the closest-pair distance $\delta'$ found over all pairs of points in $Y'$
            \item If $\delta' < \delta$, then the verical strip does indeed contain a closer pair than the recursive calls found. Return this pair and its distance $\delta'$. Otherwise, return the closest pair and its distance $\delta$ found by the recursive calls.
        \end{enumerate} 
    \end{itemize}
\end{itemize}
\textbf{Correctness}
\begin{itemize}
    \item By bottoming out the recursion when $|P| \leq 3$, we ensure that we never try to solve a subproblem consisting of only one point. 
    \item We need only to check the 7 points following each point \textit{p} in array \textit{Y'}; suppose that at some level of the recursion, the closest pair of points is $p_L \in P_L$ and $p_R \in P_R$. Thus, the distance $\delta'$ between $p_L$ and $p_R$ is strictly less than $\delta$. Point $p_L$ must be on or to the left of line \textit{l} and less than $\delta$ units away. Similarly, $p_R$ is on or to the right of \textit{l} and less than $\delta$ units away. Moreover, $p_L$ and $p_R$ are within $\delta$ units of each other vertically. Thus, $p_L$ and $p_R$ are within a $\delta \times 2\delta$ rectangle centered at \textit{l}.
    \item We next show that at most 8 points of \textit{P} can reside within this $\delta \times 2\delta$ rectangle. Consider the $\delta \times \delta$ square forming the left half of this rectangle. Since all points within $P_L$ are at least $\delta$ units apart, at most 4 points can reside within this square. Similarly, at most 4 points in $P_R$ can reside within the $\delta \time \delta$ square forming the right half of the rectangle. Thus, at most 8 points of \textit{P} can reside within $\delta \times 2 \delta$ rectangle.
    \item Now, we can see why we need to check only the 7 points following each point in the array \textit{Y'}. Still assuming that the closest pair is $p_L$ and $p_R$, let us assume that $p_L$ precedes $p_R$ in array \textit{Y'}. Then, $p_R$ is in one of the 7 positions following $p_L$. Thus, we have shown the correctness of the closest-pair algorithm.
\end{itemize}
\end{document}